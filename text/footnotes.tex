\sepfootnotecontent{fn:local-martingale}{For completeness, a stochastic process $\{ X_t \}$ is called a local martingale if there exists a sequence of stopping times $\{\tau_n \mid n \geq 1 \}$, almost surely increasing to infinity, such that the stopped process $X_{t \wedge \tau_n}$ is a martingale for every $n \geq 1$.}

\sepfootnotecontent{fn:kendall}{In queuing theory, it is standard to identify models using Kendall's notation, where the first letter identifies the distribution of the inter-arrival times, the second that of the service times, and the final entry gives the number of servers. An M/G/1 queue therefore has a single server, exponentially distributed (`Markovian') inter-arrival times, and its service times follow a general distribution.}

\sepfootnotecontent{fn:indicator}{In a slight abuse of notation, we will take $\mathbbm{1}_{ \{ W_t > 0 \} }$ to mean $\omega \mapsto \mathbbm{1}_{ \{ t \geq 0 \mid W_t(\omega) > 0 \} }$. This is consistent with \cite{kyprianou} and avoids the uglier alternative of $\mathbbm{1}_{\{ W_{\cdot} > 0 \}}$.}

\sepfootnotecontent{fn:laplace}{Or inverse transform, depending on the convention used.}